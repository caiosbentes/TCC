%pacote para gerar tabelas
\usepackage[table,xcdraw]{xcolor}
\usepackage{graphicx,url}
\usepackage{amsmath,amssymb}
\usepackage[brazil]{babel}
\usepackage[utf8]{inputenc} %For Linux users
\usepackage{appendix}
% \usepackage[none]{hyphenat}
\usepackage{mathrsfs} %math alphabet I will use for sets
\usepackage{makeidx}  %to generate indices, I guess
\usepackage{color,amsmath,cite,url,amssymb}
\usepackage{multirow} %tables with multiple rows
\usepackage{listings} % to list source code: http://www.usq.edu.au/users/leis/notes/latex/code.html
\usepackage{pdflscape}
\usepackage{graphicx}
\usepackage[bf, hang]{caption}
\usepackage{subcaption}
\usepackage{hyperref}

% para criar a lista de siglas
\usepackage{nomencl}
\makenomenclature

%  para não iniciar a contar as referências inseridas em títulos de ilustrações
\usepackage{template/notoccite}

\usepackage[chapter, Algoritmo]{algorithm}

%\figures in latex
\usepackage{tikz}
\usetikzlibrary{shapes,arrows}

% \usepackage[num]{abntcite}

% \lstset{language=matlab}
\lstset{showstringspaces=false} % no special string spaces
\lstset{identifierstyle=} % nothing happens
\lstset{keywordstyle=} % nothing happens
\lstset{linewidth=\textwidth}  %framed box is the text size
\lstset{frame=trbl}
\lstset{basicstyle=\small} % print whole listing small
\lstset{firstnumber=1, numberfirstline=false, numbers=left, numberstyle=\tiny, stepnumber=5, numbersep=5pt} %add 
\usepackage{ifpdf} %The package provides the switch \ifpdf:
\makeindex
% Glossary
%incluir pdf
\usepackage{pdfpages}


%%%%%%%%%%%%%%%%%%%%%
\newcommand{\vs}{\textit{vs}}
\newcommand{\ingles}[1]{\textcolor{red}{termo em inglês} \textit{#1}}
\newcommand{\etal}{\textit{et~al.}}
\newcommand{\eg}{\textit{e.g.,}~}

\renewcommand{\listalgorithmname}{Lista de Algoritmos}
%%%%%%%%%%%%%%%%%%%%%