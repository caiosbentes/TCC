\chapter{Introdução \label{cap:intro}}

Em ambientes de acesso restrito, como pequenas e grandes empresas, organizações, ins\-ti\-tui\-ções públicas e também em ambientes residenciais, o uso de técnicas de controle de acesso físico é essencial para garantir a privacidade do local e a segurança de pessoas e bens \cite{fennelly2012effective}. O controle de acesso  é um termo da área de segurança que faz referência à prática de permitir o acesso a um determinado local, como um prédio ou uma sala, por exemplo, apenas para pessoas autorizadas. 

O controle físico de acesso pode ser obtido através de pessoas (um guarda, segurança ou recepcionista) \cite{hess2008introduction, tyska2000physical}; através de meios mecânicos como fechaduras e chaves; ou através de meios tecnológicos, como sistemas baseados em soluções eletrônicas de controle de acesso. O controle realizado por pessoas ou através de meios mecânicos pode resultar na ocorrência de situações indesejadas, como, por exemplo, o acesso de visitantes não autorizados, falta de controle do número de pessoas e ineficiência ou ausência do controle de horário de entrada e saída. Desta forma, é necessário a automatização do controle de acesso através do uso de tecnologias, como a Identificação por Rádio-Frequência (\textit{Radio-Frequency IDentification} -- RFID) ou a biometria, que serão abordadas neste trabalho. 

\nomenclature{RFID}{Identificação por Rádio-Frequência (\textit{Radio-Frequency IDentification})}


Este trabalho apresenta uma proposta de solução de controle de acesso físico utilizando tecnologia de biometria por impressão digital. Este sistema é baseado em \textit{hardware} e \textit{software} livre, como a plataforma de prototipagem eletrônica Arduino e a plataforma Node.js, utilizada para o desenvolvimento de aplicações web \textit{server-side}. Desta forma, o sistema proposto é composto por um aparelho biométrico e uma plataforma web utilizada para o gerenciamento do controle de acesso de usuários.



\section{Objetivos}
Nesta seção são apresentados os objetivos, geral e específicos, definidos para o desenvolvimento deste trabalho.

\subsection{Geral\label{section:objetivo_geral}}

 O objetivo deste trabalho é propor uma solução de controle de acesso físico de custo reduzido, utilizando recursos de \textit{hardware} e \textit{software} livre. Assim, deve ser implementado um sistema eletrônico composto por um dispositivo de \textit{hardware}, com tecnologia de biometria por impressão digital; e um servidor, para o armazenamento e gerenciamento de dados. A principal finalidade da solução proposta aqui é proporcionar segurança às pessoas e ao seu patrimônio, impedindo a entrada de pessoas não autorizadas ao ambiente controlado. Portanto, esse sistema deve ser capaz de limitar e registrar a entrada de pessoas em locais de acesso restrito, como salas de trabalho ou laboratórios, por exemplo. Desta forma, deve ser elaborado um projeto de \textit{hardware} para definir os recursos que serão utilizados, bem como o sistema embarcado, componentes, sensores e as funcionalidades do aparelho; e um projeto de \textit{software} para definir a estrutura do servidor, banco de dados, plataforma web e as ferramentas necessárias para implementação desse sistema.


\subsection{Específicos}                        

A partir do objetivo geral descrito na Seção~\ref{section:objetivo_geral} são estabelecidos a seguir os objetivos específicos a serem alcançados na apresentação deste trabalho e no desenvolvimento da solução proposta.

\begin{itemize}
   
   
  \item  Apresentar uma introdução sobre as principais tecnologias envolvidas no desenvolvimento da solução;

  \item  Apresentar trabalhos acadêmicos e produtos de mercado relevantes, citando soluções eletrônicas pertinentes à área de controle de acesso físico;
  
  \item Elaborar um pré-projeto para definir as funcionalidades do sistema e os requisitos de \textit{hardware} e \textit{software} necessários para a sua implementação. Definir a técnica de controle de acesso a ser adotada, sistema embarcado, componentes eletrônicos, ferramentas de programação, projeto de circuito, linguagem de programação, servidor e banco de dados;
  
  \item Elaborar o projeto de um circuito para a integração dos componentes de \textit{hardware}; confeccionar a placa de circuito impresso a partir do projeto do circuito; montar o circuito; confeccionar uma carcaça para embutir o aparato eletrônico;

  \item Implementar um \textit{software} para o funcionamento do sistema embarcado; criar um módulo servidor com banco de dados para cadastrar usuários, armazenar dados de acesso (nome, data e hora de acesso), e trocar informações de controle com o sistema embarcado responsável pelo controle de uma fechadura eletrônica; criar uma plataforma web para o gerenciamento de usuários e emissão de relatórios de acesso;
  
  \item Implantar o sistema, testar, validar suas funcionalidades, avaliar os resultados durante a sua operação e propor novos avanços.
  
\end{itemize}



\section{Justificativa}



A implementação do sistema proposto aqui justifica-se pela necessidade, por parte da Universidade Federal do Pará (UFPA), de um sistema de custo reduzido que substitua o uso de métodos tradicionais, como por exemplo, chaves, senhas e cartões, para controlar a entrada e saída de pessoas em locais de acesso restrito. Os métodos tradicionais são relativamente inseguros, pois não  exigem autenticação e podem ser  facilmente  utilizados por terceiros, já que há a possibilidade de serem roubados, perdidos ou emprestados, e a partir disso forjados ou copiados. Além disso, esses métodos não possuem nenhum tipo de registro de acesso ou gerenciamento de usuários. Assim, surgem tecnologias que se mostram vantajosas quanto a esse tipo de problema, a principal delas e uma das mais seguras é a biometria digital \cite{newman2009security}, uma forma de controle de acesso autenticado que proporciona maior segurança e praticidade, pois o elemento que permite a entrada do usuário é a sua própria identidade biométrica, uma característica única de cada indivíduo.
 
\nomenclature{UFPA}{Universidade Federal do Pará}

Dentre as tecnologias de leitura de biometria atualmente estudadas, a impressão digital é uma das mais populares, além de apresentar menor custo de implantação em relação as demais, como mostra. O custo de desenvolvimento das aplicações de biometria utilizando impressão digital é um aspecto importante a ser considerado, pois os sistemas oferecidos no mercado ainda apresentam um custo elevado de aquisição, manutenção e suporte. Por isso, uma das estratégias adotadas neste projeto é o desenvolvimento de uma solução empregando recursos de \textit{hardware} e \textit{software} livre, pois esse tipo de recursos beneficia o desenvolvedor e o cliente, tanto com relação aos custos quanto ao suporte oferecido por esse tipo de ferramenta.


\section{Organização do Trabalho}

Este trabalho é dividido em seis capítulos. No Capítulo~\ref{cap:intro} são apresentados a introdução, objetivos, justificativa e organização deste trabalho. Os demais capítulos apresentam a seguinte organização: 

O Capítulo~\ref{cap:fundamentacao} aborda as principais tecnologias presentes em cada uma das etapas de implementação do sistema proposto neste trabalho. Desta forma, o capítulo supracitado aborda conceitos sobre técnicas de controle de acesso, biometria, sistemas embarcados, projeto de circuitos eletrônicos, desenvolvimento de aplicações de rede e banco de dados. 

O Capítulo~\ref{cap:estadodaarte} é dividido em duas seções: Soluções acadêmicas (Seção~\ref{solucoes_academicas}), a qual aborda algumas propostas acadêmicas de controle de acesso utilizando cartões eletrônicos de identificação, impressão digital e multibiometria; e Soluções de mercado (Seção~\ref{solucoes_mercado}), que apresenta os principais produtos eletrônicos de controle de acesso físico comercializados no mercado de segurança.


O Capítulo~\ref{cap:desenvolvimento} apresenta a solução proposta neste trabalho para controle de acesso físico utilizando tecnologia de biometria por impressão digital e uma aplicação de gerenciamento online.

O Capítulo~\ref{cap:resultados} é apresentado em duas grandes seções: Seção~\ref{testes&resultados_aparelho} e \ref{testes&resultados_servidor}. Nessas seções são descritos os testes realizados sobre o aparelho Setfinger e o módulo servidor, além dos resultados obtidos a partir desses testes.

Por fim, no Capítulo~\ref{cap:conclusao} são apresentadas as conclusões acerca deste trabalho, e algumas sugestões para trabalhos futuros.



