\chapter{Testes e Resultados\label{cap:resultados}}
 Neste Capítulo são relatados os testes realizados sobre o aparelho Setfinger e o módulo servidor. Os objetivos dos testes de \textit{hardware} são: avaliar o desempenho da MCU e do Controlador Ethernet; verificar a eficiência do circuito de alimentação; verificar as conexões via cabos e trilhas para identificar e corrigir possíveis falhas elétricas; e confirmar o funcionamento de cada componente através destes unitários. Os objetivos dos testes de \textit{software} são: avaliar a comunicação do servidor TCP com o aparelho Setfinger, avaliar o desempenho do banco de dados Mysql, verificar a compatibilidade dos \textit{frameworks}, utilizados no módulo servidor, com diferentes sistemas operacionais, e validar as funcionalidades da página de gerenciamento e verificar o conteúdo do banco de dados.
 
 
 
\section{Aparelho Setfinger\label{testes&resultados_aparelho}}

Além do \textit{hardware} proposto, o sistema Setfinger foi testado em conjunto com as versões de \textit{hardware} 1.0 e 2.0 apresentadas no Apêndice~\ref{hardware_1.0} e~\ref{hardware_2.0}, respectivamente. Todas essas versões foram instaladas e testadas no CTIC, e permanecem sendo utilizadas para o propósito ao qual foram destinadas. Desta forma, são apresentados a seguir os testes realizados sobre o \textit{hardware} Setfinger e os seus respectivos resultados.


\begin{itemize}
    \item O aparelho Setfinger foi mantido ligado durante um período de no mínimo 30 dias, 24 horas por dia, e não apresentou problemas com o uso contínuo.
    
    \item Foi realizado teste unitário para verificar a precisão de leitura das teclas do \textit{keypad} com o circuito \textit{one wire kaypad}. Além disso, foi realizado um teste de desgaste das teclas, no qual verificou-se o funcionamento das teclas após serem pressionadas cerca de 100 vezes. O circuito \textit{one wire kaypad} utilizado para a redução de fios de conexão com o Arduino não apresentou o resultado esperado, pois ocorreram erros na leitura das teclas pressionadas. Além disso, notou-se que a leitura correta das respectivas teclas do dispositivo \textit{keypad} ocorre somente quando esse dispositivo é alimentado na saída $5$ V fornecido pelo Arduino. Quando, alimentado na saída $5$ V fornecido pelo regulador extra L7805, as teclas pressionadas são atribuídas a caracteres não correspondentes aos quais elas representam. Por exemplo, quando a tecla ``1'' é pressionada, o aparelho Setfinger a reconhece como a tecla ``2''. Portanto, propõe-se a substituição do circuito \textit{one wire kaypad} por uma nova abordagem utilizando o circuito integrado expansor de portas PCF8574AP Philips/NXP \cite{pcf8574}, baseado em tecnologia I2C. Esse Circuito Integrado (CI) possibilita a redução do numero de conexões do teclado ao Arduino, pois necessita apenas de duas vias de dados (SDA e SCL) e duas vias de alimentação ($5$ V e GND).

    \nomenclature{CI}{Circuito Integrado}
    
    
    \item Foram gravadas apenas as impressões digitais de 15 usuários no leitor Fingerprint, devido a falta de voluntários. O fabricante do sensor Fingerprint informa em sua documentação que a memória \textit{flash} desse dispositivo tem capacidade para o armazenamento de apenas 162 modelos de impressões digitais (\textit{templates}) \cite{zfm-20, zfm20ada}. No entanto, foi realizado um teste, no qual as impressões digitais de um mesmo usuário foram gravadas além de 162 vezes. Desta forma, para cada gravação foi atribuído um ID sequencialmente, de 1 a 200. O \textit{datasheet} do sensor Fingerprint ZFM-20 não fornece informações que possibilitem a compreensão lógica de gravação dos modelos de impressão digital. Além disso, a quantidade de usuários cadastrados nesse teste é insuficiente para verificar o comportamento do sensor mediante a gravação do número limite de impressões digitais recomendado pelo fabricante. Os processos internos de gravação e leitura do sensor Fingerprint não são objetos de estudo deste trabalho. Portanto, não foi possível concluir o que ocorre após a gravação de 162 modelos digitais. Empiricamente, as possibilidades são que o sensor sobrescreva aleatoriamente algum espaço de memória ocupado por um modelo digital ou que ele possua um espaço de memória suficiente para gravar um número maior que 162 \textit{templates}. 
    
    Quanto aos resultados de sua utilização, o sensor Fingerprint apresentou um bom desempenho no reconhecimento das impressões digitais dos usuários. No entanto, a gravação da digital dos usuários deve ser realizada de forma adequada, caso contrário o sensor apresenta dificuldades para o reconhecimento da impressão digital do usuário. Desta forma, no momento da gravação da impressão digital, preferencialmente com o dedo limpo, o usuário deve manter o seu dedo corretamente pressionado contra a lente óptica do sensor no momento da captura da imagem da sua impressão digital.
    
    \item Foi realizada uma verificação de temperatura sobre os reguladores L7805, L7809 e AMS 1117, a fim de detectar a ocorrência de possíveis sobreaquecimentos que podem danificar os componentes do \textit{hardware} Setfinger. Através desse teste observou-se um aquecimento razoável sobre os reguladores citados. No entanto, não houve sobreaquecimento e nenhuma ocorrência de defeitos eletrônicos devido a excessos de temperatura.
     
\end{itemize}



\section{Módulo Servidor (Setsever)\label{testes&resultados_servidor}}

A seguir são apresentados os testes realizados sobre o servidor TCP, banco de dados e plataforma web, e os resultados obtidos a partir da aplicação desses testes. 


\begin{itemize}
    
  \item O banco de dados e a plataforma web dependem de um servidor web. Desta forma, foram instalados os servidores web LEMP (Linux Nginx MySQL PHP) e WAMP (Windows Apache Mysql PHP), nos sistemas operacionais Linux e Windows, respectivamente. No Windows Seven foi instalado e utilizado a plataforma de desenvolvimento web WAMP. No Linux o servidor Setserver foi testado em duas distribuições diferentes, Debian 8 e Mint 17. Em ambas as versões foi utilizado a plataforma LEMP. O servidor web WAMP foi facilmente utilizado, não apresentando erros durante o processo de instalação, tampouco durante a execução dos seus serviços. No Mint 17 a instalação foi realizada com sucesso, sem a ocorrência de erros. No entanto, no Debian 8 houve uma grande dificuldade na instalação do \textit{Phpmyadmin}. 
  \nomenclature{WAMP}{Windows Apache Mysql PHP}
  \nomenclature{LEMP}{Linux Nginx MySQL PHP}
  
  \item O Servidor TCP é baseado em node.js, por isso depende da plataforma node.js, a qual possui versões para ambos os sistemas operacionais supracitados. Portanto, o node.js foi instalado nos sistemas operacionais Windows Seven, Debian 8 e Mint 17, para a execução do \textit{software} servidor TCP. Além disso, os comandos utilizados em interpretadores de linha de comandos, para a execução de \textit{software} baseados em node.js ocorre de maneira semelhante em todas as suas versões. O \textit{software} servidor TCP funcionou normalmente em todos os sistemas operacionais em que o node.js foi instalado, apresentando comportamento semelhante em todos os testes realizados.  
  
  \item A plataforma web foi hospedada nos servidores web LEMP e WAMP. No WAMP a plataforma web Setfinger foi testada apenas no \textit{localhost}. No entanto, no LEMP a plataforma web foi testada com um domínio público. Através da plataforma web foram cadastrados 15 usuários e registrados mais de 1000 acessos de usuários. Além disso, o banco de dados Mysql apresentou um bom desempenho e as consultas e inserções realizados pelo servidor TCP e pela plataforma web, funcionaram normalmente.   
  
  


\end{itemize}


